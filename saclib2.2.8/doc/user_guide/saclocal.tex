\documentclass{article}
\usepackage{saclib}

%%%%%%%%%%%%%%%%%%%%%%%%%%%%%%%%%%%%%%%%%%%%%%%%%%%%%%%%%%%%%%%%%%%%%%%%%%%

\begin{document}

%%%%%%%%%%%%%%%%%%%%%%%%%%%%%%%%%%%%%%%%%%%%%%%%%%%%%%%%%%%%%%%%%%%%%%%%%%%
\title{
Addendum\\to the\\\saclib\ User's Guide\thanks{
  \saclib\ version 1.0, ``Addendum'' version 1.0.
}
}

\author{Andreas Neubacher}
\date{June 26, 1992}
\maketitle
%%%%%%%%%%%%%%%%%%%%%%%%%%%%%%%%%%%%%%%%%%%%%%%%%%%%%%%%%%%%%%%%%%%%%%%%%%%

%%%%%%%%%%%%%%%%%%%%%%%%%%%%%%%%%%%%%%%%%%%%%%%%%%%%%%%%%%%%%%%%%%%%%%%%%%%
\section{Availability}
%%%%%%%%%%%%%%%%%%%%%%%%%%%%%%%%%%%%%%%%%%%%%%%%%%%%%%%%%%%%%%%%%%%%%%%%%%%

\saclib\ is installed on \ldots in the directory ``{\tt\~{ }saclib/saclib}''.


%%%%%%%%%%%%%%%%%%%%%%%%%%%%%%%%%%%%%%%%%%%%%%%%%%%%%%%%%%%%%%%%%%%%%%%%%%%
\section{Environment}
%%%%%%%%%%%%%%%%%%%%%%%%%%%%%%%%%%%%%%%%%%%%%%%%%%%%%%%%%%%%%%%%%%%%%%%%%%%

It is recommended to define an environment variable ``{\tt \$saclib}''
containing the name of the \saclib\ directory. This can be done by adding the
line
\begin{verbatim}
  setenv saclib ~saclib/saclib
\end{verbatim}
to your ``{\tt .login}'' or ``{\tt .cshrc}'' file. \saclib\ header and
library files should then only be accessed by using this variable. If the
installation is moved to a different directory you will only have to change
the definition of this variable.

If you want to use the shell scripts provided with \saclib, the environment
variable ``{\tt \$saclib}'' is required, and you also need
\begin{verbatim}
  set path=($path $saclib/bin)
\end{verbatim}
in your ``{\tt .login}'' or ``{\tt .cshrc}'' file.


%%%%%%%%%%%%%%%%%%%%%%%%%%%%%%%%%%%%%%%%%%%%%%%%%%%%%%%%%%%%%%%%%%%%%%%%%%%
\section{Compiling and Linking}
%%%%%%%%%%%%%%%%%%%%%%%%%%%%%%%%%%%%%%%%%%%%%%%%%%%%%%%%%%%%%%%%%%%%%%%%%%%

\sloppy

The compiled library is ``{\tt \$saclib/lib/saclib.a}'', an optimized version
is ``{\tt \$saclib/lib/saclibo.a}'', a version compiled with the debug option
is ``{\tt \$saclib/lib/saclibd.a}'', and the header files are in ``{\tt
\$saclib/include}''.

Figure \ref{fMF} shows what a makefile for linking with the standard
library might look like.

\begin{figure}[htb]
\ \hrulefill\ \small
\begin{verbatim}
CFLAGS  = -DNO_SACLIB_MACROS -I${saclib}/include
LIB = ${saclib}/lib/saclib.a

OBJS = example.o

example: $(OBJS)
        ${CC} $(OBJS) $(LIB) -o example
\end{verbatim}
\ \hrulefill\ \normalsize
\caption{A sample makefile.}
\label{fMF}
\end{figure}

\fussy

%%%%%%%%%%%%%%%%%%%%%%%%%%%%%%%%%%%%%%%%%%%%%%%%%%%%%%%%%%%%%%%%%%%%%%%%%%%
\section{Additional Documentation, Tools, etc.}
%%%%%%%%%%%%%%%%%%%%%%%%%%%%%%%%%%%%%%%%%%%%%%%%%%%%%%%%%%%%%%%%%%%%%%%%%%%

In ``{\tt \$saclib/example}'' you will find sample programs which are a bit
more sophisticated than the ones given in the ``\saclib\ User's Guide''. They
should be of help in learning how to work with \saclib.

There are also two shell scripts which should facilitate searching for a
certain algorithm and displaying \saclib\ functions:
\begin{description}
\item[sdesc]
  takes a regular expression in the style of {\tt grep} and lists all
  \saclib\ functions whose description matches the regular expression.
\item[sman]
  takes a full \saclib\ function name and displays the corresponding source
  file.
\end{description}
Calling the scripts without parameters produces a usage message.

For experimenting with \saclib\ functions interactively, use ``{\tt isac}'',
which gives you a simple shell environment from which you can call all
\saclib\ functions.


%%%%%%%%%%%%%%%%%%%%%%%%%%%%%%%%%%%%%%%%%%%%%%%%%%%%%%%%%%%%%%%%%%%%%%%%%%%
\section{Where to Look for Help}
%%%%%%%%%%%%%%%%%%%%%%%%%%%%%%%%%%%%%%%%%%%%%%%%%%%%%%%%%%%%%%%%%%%%%%%%%%%

Apart from sending e-mail to the adresses given in Section 1.3 of the ``User's
Guide'' you can also ask \ldots

\end{document}

